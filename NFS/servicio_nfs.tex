\documentclass[12pt]{article}
\usepackage{url}
\begin{document}

\section{Configurar servidor NFS}

   En caso de usar m\'aquinas virtuales configurar la red por adaptador 
   puente.\\

    *Paquetes que se descargan tanto como para el servidor y el cliente\\
    \\
    1. sudo apt-get install rpcbind\\
    2. sudo apt-get update\\
    3. sudo apt-get install nfs-kernel-server\\

    \textbf{Servidor}:
    \\
    4. sudo mkdir /var/nfs/ \\
    5. sudo chown nobody:nogroup /var/nfs \\ 
    6. sudo nano /etc/exports    *se puede usar cualquier editor de texto\\
    \\
        Se agregan las lineas:\\
        /home 1.1.1.1(rw,sync,no\_root\_squash,no\_subtree\_check)   
        *1.1.1.1 ip cliente\\
        /var/nfs 1.1.1.1(rw,sync,no\_subtree\_check) \\
    \\
    7. sudo exportfs -a
    <permite exportar los cambios hechos en /etc/exports>\\
    8. sudo service nfs-kernel-server start\\

    \textbf{Cliente}:
    \\
    1.sudo apt-get update\\
    2.sudo apt-get install nfs-common\\
    3.sudo mkdir -p /mnt/nfs/home\\
    4.sudo mkdir -p /mnt/nfs/var/nfs\\
    5.sudo mount -t nfs -o resvport 2.2.2.2:/var/nfs /mnt/nfs/var/nfs/  
     \\*2.2.2.2 ip servidor\\

   En caso de ser un mac el cliente se empieza desde el \'item 3.\\


   \textbf{Comandos para desmontar servidores}:\\

   sudo umount /mnt/nfs/home\\
   sudo umount /mnt/nfs/var/nfs\\

\section{Conceptos}
        \textbf{RPCBIND}: permite que los clientes NFS descubran qué 
        puerto está utilizando el servidor NFS.\\ 
        \textbf{nfs-kernel-server}: es paquete que me permitirá 
        compartir los directorios. \\
        \textbf{rw}: Esta opción le da al equipo cliente tanto 
        opciones de lectura como de escritura en el volumen. \\
        \textbf{sync}: Esta opción NFS fuerza para escribir cambios
        en el disco antes de responder. Esto da lugar a un entorno más 
        estable y coherente, ya que la respuesta refleja el estado real del
        volumen remoto.\\
        \textbf{no\_subtree\_check}: Esta opción evita que el control 
        del subárbol, que es un proceso en el que el huésped debe comprobar
        si el archivo es en realidad todavía disponibles en el árbol 
        exportado para cada petición. En casi todos los casos, es mejor 
        desactivar el control del subárbol.\\
        \textbf{no\_root\_squash}: Por defecto, NFS traduce las 
       peticiones de un usuario root de forma remota a un usuario sin 
       privilegios en el servidor. Esto se supone que es una característica
       de seguridad al no permitir una cuenta de root en el cliente para 
       utilizar el sistema de archivos del sistema como usuario root. Esta 
       directiva desactiva esto para ciertas acciones.\\

\section{Referencias}
    https://www.digitalocean.com/community/tutorials/how-to-set-up-an-nfs-mount-on-ubuntu-14-04\\
    http://www.alcancelibre.org/staticpages/index.php/12-como-nfs\\
    http://www.linux-party.com/index.php/35-linux/8465-montar-un-directorio-remoto-via-nfs-en-linux-por-editar\\
    https://debian-handbook.info/browse/es-ES/stable/sect.nfs-file-server.html\\
    http://www.linuxito.com/nix/591-como-se-relacionan-nfsd-nfsuserd-mountd-y-rpcbind\\
    http://recursostic.educacion.es/observatorio/web/gl/software/software-general/733-nfs-sistema-de-archivos-de-red\\

\end{document}
